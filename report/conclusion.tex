\section{Conclusion}
\subsection{Evaluation of the Study}
This study has implemented the MitM approach using SCSCP, showing that the MitM 
paradigm is an achievable goal. Currently, however, the MitM server acts as 
merely a proxy, redirecting SCSCP requests to CAS that know how to evaluate them.
For MitM to be a truly modular abstract algebra environment, the following
changes must be made:
\begin{itemize}
  \item A peer-to-peer connection must be made with the CAS servers instead of a simple client-serve one, so that
    CAS servers can, in turn, query MitM if during a computation they encounter 
    a concept that lies outside their field of knowledge. In application to
    this particular case, it would be cleaner if, instead of asking MitM to 
    produce permutations of a list, the client simply queries MitM for the orbit 
    of a polynomial by defining an action of a member of the symmetric group on a 
    polynomial. GAP would then be able to calculate the orbit by making the group 
    act on the polynomial with the described action and querying MitM for 
    equality of polynomials, resulting in a linear-time algorithm instead of
    quadratic-time behaviour displayed by the current client.
  \item Alignment between CAS servers and the MitM server must be automated.
    Although manual alignment as described in this case study is usable and
    enables pinpoint alignments to be made, it is not scalable. Any CAS that aims
    to be MitM-compatible should develop a representation of its type system
    in MMT, so that, upon establishing a connection with a new CAS system, the 
    MitM server would be able to automatically align newly accessible functions 
    with symbols from public CDs. This will also enable MMT to act as more than
    a routing proxy, as it would be able to typecheck the arguments of incoming
    requests, making the MitM system more rigid.
\end{itemize}